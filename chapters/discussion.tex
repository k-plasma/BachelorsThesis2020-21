\section{Section title}

In this work, an estimated relationship between wheat crop of Hungary and its GDP (indicated with GVP) is established. The work is inspired by extensive literature reviews and some previous related studies. In some of the studies artificial neural networks (ANN) were extensively used. It has been shown that recurrent networks, specifically its variants LSTM and GRU can perform much better. This can be verified comparing the results of FFNNs and RNNs in Figure ~\ref{fig:feedforward} and Figure ~\ref{fig:trg}. 

Although the results from GRU are promising, but it poses the challenge of training the network appropriately. Specifically, how the speed and accuracy can be balanced and made optimum. With all of its pros and cons, it has been able to cater to the problems dynamically and actively. Also, these networks can be employed to solve huge sets of problems. 

From the results in Figure ~\ref{fig:trg}, it can be concluded that wheat quantity and wheat yield can significantly affect the GDP. This is particularly true since a large number of other agriproducts are directly dependent on wheat production. If the wheat yield is good enough, it can not only meet the requirements of food, but it can help flourish several related food industries. These two factors can significantly contribute to the country GDP. Also, they can help boom the exports sector of the economy. 
\subsection{Subsection title}
